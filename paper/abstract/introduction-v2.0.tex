\section{Introduction}

Over the last 20 years, much fundamental work has been done in developing high-order numerical methods for Computational Fluid Dynamics. Moreover, the need to improve and simplify these methods has attracted the interest of the applied mathematics and the engineering communities. Now, these methods are beginning to prove themselves sufficiently robust, accurate and efficient for use in real-world applications.

However, low-order numerical methods are still the standard in aeronautical industry. There has been a sustained scientific and economical investment to develop this successful and robust technology for a long time. Currently, a state-of-the-art 2nd-order Finite Volume computational tool performs adequately well in a broad range of aeronautical engineering applications. For that reason, the introduction of brand new high-order numerical schemes in the aeronautical industry is challenging, particularly in areas where the low-order numerical methods already provide the required robustness and accuracy (keeping in mind the limitations of the current turbulence model technology).

Thanks to new and emerging aircraft roles (very small or very big concepts, very high or very low altitude, quiet vehicles, low fuel consumption, etc.), revolutionary aircraft design concepts will appear in the near future and the need for high-fidelity simulation techniques to predict their performance is growing rapidly. Undoubtedly, high-order numerical methods are starting to find their place in the aeronautical industry. 

Unsteady simulation, flapping wings, wake capturing, noise prediction and Large Eddy Simulations are just a few examples of computations that require high-order numerical methods. In particular, high-order methods have a significant edge in applications that require accurate resolution of the smallest scales of the flow. Such situations include the generation and propagation of acoustic noise from the airframe, or at the limits of the flight envelope where unsteady vortex-dominated flows have a significant effect on the aircraft's performance. Utilizing a high-order representation enables the smallest scales to be represented with a greater degree of accuracy than standard second-order methods. Furthermore, high-order methods are inherently less dissipative, resulting in less unwanted interference with the correct development of the turbulent energy cascade. 

Finally, the amount of computing effort to achieve a small error tolerance can be much smaller with high-order than second-order methods. Even real time simulations (one second of computational time, one second of real flight), could benefit from high-order algorithms which count on a more intensive inner element computation (ideal for vector machines and new computational platforms like GPUs, FPGAs, coprocessors, etc).

But, before claiming the future success of the high-order numerical methods in the industry, two main difficulties should be overcome: a) high-order numerical schemes must be as robust as a state-of-the-art low-order numerical method. b) the existing level of Verification and Validation (V\&V) in high-order CFD codes should be similar to the typical level of their low-order counterparts.

During the last decade, the Aerospace Computing Laboratory (ACL) of the Department of Aeronautics and Astronautics at Stanford University has developed a series of high-order numerical schemes and computational tools, contributing massively to the demonstration of the viability of this technique. In this paper, a code called HiFiLES developed at ACL and built on top of SD++ (Castonguay et al.\cite{castonguay2011}) is described in detail with a particular emphasis on robustness in a broad range of applications and an industrial-like level of V\&V. HiFiLES takes advantage of the synergies between applied mathematics, aerospace engineering, and computer science to achieve the ultimate goal of developing an advanced high-fidelity simulation environment.

Apart from the original characteristics of the SD++ code (described in Castonguay et al.\cite{castonguay2011}), HiFiLES includes some important physical models and computational methods such as: Large Eddy Simulation (LES) using explicit filters and advanced subgrid-scale (SGS) models, high-order stabilization techniques, shock detection and capturing for compressible flow calculations, convergence acceleration methodologies like p-multigrid, and local and dual time stepping.

During the development of this software several key decisions have been taken to guarantee a flexible and lasting infrastructure for industrial Computational Fluid Dynamics simulations:
\begin{itemize}
\item The selection of the Energy-Stable Flux Reconstruction (ESFR) scheme on unstructured grids. The flexibility of this method has been critical to guarantee a correct solution independently of the particular physical characteristics of the problem.
\item High performance, materialized in a multi-GPU implementation which takes advantage of the ease of parallelization afforded by discontinuous solution representation. Furthermore, HiFiLES aims to guarantee compatibility with future vector machines and revolutionary hardware technologies.
\item Code portability by using ANSI C++ and relying on widely-available, and well-supported mathematical libraries like Blas, LAPACK, CuBLAS and ParMetis.
\item Object oriented structure to boost the re-usability and encapsulation of the code. This abstraction enables modifications without incorrectly affecting other portions of the code. Although some level of performance is traded for re-usability and encapsulation, the loss in performance is minor.
\end{itemize}

As mentioned before, the mathematical basis and computational implementation of HiFiLES were described in Castonguay et al. For that reason, the goal of this paper is to illustrate the level of robustness of HiFiLES in complex problems, via a detailed Verification and Validation study which is fundamental to increase the credibility of this technology in a competitive industrial framework.

In particular, to ensure that the implementation of the aforementioned features in HiFiLES is correct, the following verification tests will be shown in the final version of this paper: checks of spatial and temporal order of accuracy using the Method of Manufactured Solutions (MMS) in 2D and 3D for viscous and inviscid flows, characterization of stable time-step limits, assessment of computational cost per degree of freedom for a given error tolerance, and measurement of weak and strong scalability in GPUs and CPUs. 

After the Verification, a detailed Validation of the code will be presented to illustrate that the solutions provided by HiFiLES are an accurate representation of the real world. Simulations of complex flows will be validated against experimental or Direct Numerical Simulation (DNS) results for the following cases: laminar and turbulent flat plane, flow around a circular cylinder, subsonic flow attached in a NACA0012, SD7003 wing-section and airfoil at 4$\degr$ angle of attack, LES of the Taylor-Green Vortex, and DNS and LES of Decaying Homogeneous Turbulence.

The organization of this paper is as follows. Section \ref{sec:govEq} provides a description of the governing equations. Section \ref{sec:numerics} describes the mathematical numerical algorithms implemented in the code (with a particular emphasis in convergence acceleration techniques and stability). \ref{sec:verification} and \ref{sec:validation} focus on the Verification and Validation of HiFiLES, and finally, the conclusions are summarized in section \ref{sec:conclusion}.

Finally, it is our intent for this paper to be the main reference for work that uses or enhances the capabilities of HiFiLES, and for it to serve as a sort of reference for researchers and engineers that would like to develop or implement High-order numerical schemes based on an Energy-Stable Flux Reconstruction (ESFR) scheme. 