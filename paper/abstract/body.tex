\section{Flux Reconstruction: Viscous case (1D)}
\begin{raggedleft}
Now consider solving the following 1D diffusion equation,
\end{raggedleft}
\begin{equation}
\frac{\partial u}{\partial t} - \frac{\partial f_{v}}{\partial x} = 0,
\label{governing}
\end{equation}
%
where $f_{v}$ is now the viscous flux. This flux is (at the very least) a function of the solution gradient. In general, the viscous flux takes the following form,
%
\begin{equation}
f_{v} = f_{v} \left(\lambda, u, \frac{\partial u}{\partial x} \right).
\end{equation}
%
Here, $\lambda$ is a measure of the medium's viscosity. On the standard element (defined previously), this model equation takes the form,
%
\begin{equation}
\frac{\partial\hat{u}^{\delta D}}{\partial t} = \frac{\partial\hat{f_{v}}^{\delta}}{\partial r},
\end{equation}
%
where the continuous flux of degree $k+1$ is, 
%
\begin{equation}
\hat{f_v}^{\delta}=\hat{f_v}^{\delta}(r,t)=\frac{f^{\delta}_{v n}(\Gamma_n^{-1}(r),t)}{J_n}.
\end{equation}
%
The continuous flux $\hat{f_{v}}^{\delta}$ must be constructed based on solution information from the current element and its nearest neighbors. This information is discontinuous at the element boundaries. On the standard element, the discontinuous solution and the discontinuous solution gradient can be written as,
%
\begin{equation}
\hat{u}^{\delta D}=\sum_{i=0}^{k}\hat{u}^{\delta D}_{i}\;l_i,		\qquad \qquad		\frac {\partial \hat{u}^{\delta D}}{\partial r}=\sum_{i=1}^{k}\hat{u}^{\delta D}_{i}\; \frac{\partial l_i}{\partial r}.
\end{equation}
%
Where the solution and its gradient are polynomials of order k and k-1 respectively. The flux based on these discontinuous quantities will be discontinuous. The standard approach involves correcting the solution and its gradient such that they assume continuous values at the boundary, thereby ensuring a continuous viscous flux. The FR approach (presented in the previous section) provides a clear and intuitive process for treating discontinuities in the inviscid flux. The same process can be applied to treat discontinuities in the solution and the solution gradient. The solution within each element represents a $k^{th}$ order discontinuous function which can be corrected by a $(k+1)^{th}$ order function to achieve a continuous form. In 1D, the $(k+1)^{th}$ order correction function can be composed from $(k+1)^{th}$ order left and right correction functions multiplied by jumps in the solution at the left and right boundaries respectively. The correction to the solution would assume the form,
%
\begin{equation}
\hat{u}^{\delta C}=(\hat{u}^{\delta I}_L-\hat{u}^{\delta D}_L)g_L+(\hat{u}^{\delta I}_R-\hat{u}^{\delta D}_R)g_R.
\end{equation}
%
Here, $\hat{u}^{\delta I}_L$ is the common value of the solution at the left interface and $\hat{u}^{\delta I}_R$ is the common value at the right interface. This approach is directly analogous to the process of correcting the $k^{th}$ order discontinuous flux with the $(k+1)^{th}$ order correction functions (see equation 15). The correction functions for the solution and the inviscid flux must have similar characteristics, attaining the value of 1 at their respective boundary, and zero at the opposite boundary. In addition, to preserve accuracy, the correction functions must be the same order (k+1 for both the solution and the inviscid flux) \cite{Huynh09}. Thus, for simplicity, the flux correction functions can be used to correct both the inviscid flux and the solution. The continuous solution would then assume the form,
%
\begin{equation}
\hat{u}^{\delta}=\hat{u}^{\delta D}+\hat{u}^{\delta C}=\hat{u}^{\delta D}+(\hat{u}^{\delta I}_L-\hat{u}^{\delta D}_L)g_L+(\hat{u}^{\delta I}_R-\hat{u}^{\delta D}_R)g_R,
\end{equation}
%
where the correction functions are now the same as defined in the previous section. The formulation of the continuous solution allows for straightforward computation of the continuous solution gradient using,
%
\begin{equation}
\frac {\partial \hat{u}^{\delta}}{\partial r} = \sum_{i=1}^{k}\hat{u}^{\delta D}_{i}\; \frac{\partial l_i}{\partial r} + (\hat{u}^{\delta I}_L-\hat{u}^{\delta D}_L) \frac {\partial g_L}{\partial r} + (\hat{u}^{\delta I}_R-\hat{u}^{\delta D}_R) \frac {\partial g_R}{\partial r},
\end{equation}
%
where the continuous solution gradient is a polynomial of order k, derived from the continuous solution of order k+1 (see equation 29). The continuous solution at the boundary can be determined by evaluation of equation 29 at the left or right boundary. By construction, when the continuous solution is evaluated at the boundary, one obtains the expected common solution value at the boundary. If the BR2 approach is followed, this is simply the arithmetic average of the left and right solution states. At the right boundary of element $j$,
%
\begin{equation}
\hat{u}^{\delta I}_R =  \frac {1}{2} \left ( \hat{u}^{\delta D}_R |_{j} + \hat{u}^{\delta D}_L |_{j+1}  \right),
\end{equation}
%
where $\hat{u}^{\delta D}_R |_{j} $ is the solution at the right interface of element j. The value of the solution gradient at the boundary is more interesting. During the solution gradient's construction, it acquires components of the correction function gradients. The common gradient at the boundary is simply equation 30 averaged and evaluated at the boundary. The resulting form at the right boundary of element $j$ is,
%
\begin{equation}
\left (\frac {\partial \hat{u}}{\partial r} \right)_{R}^{\delta I} =  \frac {1}{2} \left( \frac {\partial \hat{u}^{\delta D}_R |_{j}}{\partial r} +  \frac {\partial \hat{u}^{\delta D}_L |_{j+1}}{\partial r} \right) + \frac{1}{2} \left[ (\hat{u}^{\delta I}_R-\hat{u}^{\delta D}_R) |_{j} \frac {\partial g_{R}}{\partial r} (1) + (\hat{u}^{\delta I}_L-\hat{u}^{\delta D}_L) |_{j+1} \frac {\partial g_L}{\partial r} (-1) \right] .
\end{equation}
%
Further simplifying this expression yields,
%
\begin{equation}
\left( \frac {\partial \hat{u}}{\partial r} \right)_{R}^{\delta I} =  \frac {1}{2} \left( \frac {\partial \hat{u}^{\delta D}_R |_{j}}{\partial r} +  \frac {\partial \hat{u}^{\delta D}_L |_{j+1}}{\partial r} \right) - \frac{1}{4} \left[ \frac {\partial g_{R}}{\partial r} (1) -  \frac {\partial g_L}{\partial r} (-1)  \right]  \left( \hat{u}^{\delta D}_R |_{j} - \hat{u}^{\delta D}_L |_{j+1}  \right) .
\end{equation}
%
This form compares favorably to the BR2 formulation, with the lifting function term replaced with a term involving the gradients of the correction functions.  In the BR2 formulation, the common solution gradient is formed as the average of the gradient at the interface penalized (or corrected) by the jump in the solution at the boundary. 
%
\begin{equation}
\left( \frac {\partial \hat{u}}{\partial r} \right)_{R}^{\delta I} =  \frac {1}{2} \left( \frac {\partial \hat{u}^{\delta D}_R |_{j}}{\partial r} +  \frac {\partial \hat{u}^{\delta D}_L |_{j+1}}{\partial r} \right) - \eta^{e} * r_e\left( \hat{u}^{\delta D}_R |_{j} - \hat{u}^{\delta D}_L |_{j+1}  \right)
\end{equation}
%
Here, $\eta^{e}$ is a weighting constant and $r_e$ is a lifting operator (a function) defined based on each edge (e) of the element. The similarity between equations 33 and 34 is further evidence that standard viscous terms in the traditional DG framework have counterparts in the differential FR framework. This result is not new, and similar assertions have been made by Huynh and Gao \cite{Huynh09} \cite{Gao09}. However, there remain a number of unanswered questions and concerns regarding the formulation of viscous terms in the FR framework. These concerns are summarized by the following queries:

\vspace{0.2in}
\begin{raggedleft}
{\bf Query 1:} Given a common viscous flux at the boundary, what is the optimal way to update the discontinuous viscous flux at the solution points?
\end{raggedleft}
\vspace{0.2 in}

The common solution and solution gradient at the boundary uniquely specify the viscous flux at the boundary. However, this common flux is generally inconsistent with the flux at the interior solution points. Computing the common viscous flux is analogous to computing the common inviscid flux. Recall that in the inviscid case, the common flux was generally not the same as the the discontinuous flux on either side of the interface. The difference between the discontinuous flux and the common flux was multiplied by correction functions and used to correct the flux at the interior solution points (see equation 16). This procedure produced a continuous flux at the solution points. This continuous inviscid flux was now in agreement with the common inviscid flux at the boundary. The same process can be utilized to force agreement between the discontinuous viscous flux at the solution points and the common viscous flux at the boundary. In this process, the continuous viscous flux is constructed using the following steps ...

\vspace{0.1 in}
\noindent \emph{Scheme A}  	
\vspace{0.1 in}

\begin{enumerate}
\item Reconstruct the common solution and the solution gradient at the boundary using equations 31 and 33
\item Construct the common viscous flux at the boundary using the common solution, common gradient, and the analytical form of the flux $f(u, \frac {\partial u}{\partial x})$
\item Compute the discontinuous viscous flux at the solution points using the analytical form for $f(u, \frac {\partial u}{\partial x})$
\item Compute the discontinuous viscous flux at the boundary using the analytical form for $f(u, \frac {\partial u}{\partial x})$
\item Compute corrections: the differences between the continuous and discontinuous fluxes at the boundary
\item Use the corrections and the correction functions to update the discontinuous flux at the solution points; thereby obtain the continuous flux
\end{enumerate}

\vspace{0.1 in}

\noindent The viscous flux can thus be computed using a reconstruction procedure applied to the solution at the boundary, and a similar procedure applied to the entirety of the viscous flux. In this approach (call it Scheme A), the solution reconstruction at the boundary serves the purpose of creating a common viscous flux at the interface. The viscous flux at the solution points is indirectly corrected using this common flux and the correction functions (of degree k+1) associated with the standard FR procedure. In this scenario, the corrective flux takes the form,
\begin{equation}
\hat{f_v}^{\delta C}=(\hat{f_v}^{\delta I}_L-\hat{f_v}^{\delta D}_L)g_L+(\hat{f_v}^{\delta I}_R-\hat{f_v}^{\delta D}_R)g_R,
\end{equation}
and the final continuous flux becomes,
\begin{equation}
\hat{f_v}^{\delta}=\hat{f_v}^{\delta D}+\hat{f_v}^{\delta C}=\hat{f_v}^{\delta D}+(\hat{f_v}^{\delta I}_L-\hat{f_v}^{\delta D}_L)g_L+(\hat{f_v}^{\delta I}_R-\hat{f_v}^{\delta D}_R)g_R,
\end{equation}

\noindent where the continuous viscous flux is reconstructed in a procedure identical to that of the inviscid flux (see equations 15 and 16).

\vspace{0.1 in}

\noindent As an alternative course, one may undertake an approach which eschews any form of flux reconstruction. Consider an approach (call it Scheme B) where only the solution and the solution gradient are reconstructed. The resulting continuous solution and solution gradient can be used to compute the viscous flux at the solution points. This viscous flux will be consistent with the viscous flux at the boundary (as both quantities are computed based on the same continuous reconstructed solution and solution gradient). The steps in this alternative algorithm are as follows: 

\vspace{0.1 in}
\noindent \emph{Scheme B}	
\vspace{0.1 in}
\begin{enumerate}
\item Reconstruct the solution at the boundary using equation 31
\item Reconstruct the solution and the solution gradient at the solution points using equations 29 and 30
\item Compute the viscous flux at the solution points using the analytical form of $f(u, \frac {\partial u}{\partial x})$
\end{enumerate}

\vspace{0.1 in}

\noindent Thus, in Scheme B, the solution and solution gradient are reconstructed at the solution points and the boundary. In the former approach (Scheme A), the solution and solution gradient are reconstructed at the boundary only and the flux is reconstructed at the solution points. The distinction between the approaches of Scheme A and Scheme B may be important from an accuracy standpoint. It is possible that scheme B will suffer a loss of accuracy by neglecting to reconstruct the flux with a higher order (k+1) degree polynomial. For Scheme B's approach, reconstructing the solution and the solution gradient to order k+1 and k respectively may not ensure that the flux attains order k+1. However, it is also possible that both schemes A and B yield the same order of accuracy, and that the latter is less computationally costly than the former. The distinctions between these two approaches will be numerically explored through the examination of model diffusion problems presented later in this paper.

\vspace{0.1 in}
\noindent \emph{Scheme C}
\vspace{0.1 in}

\noindent A third 'hybrid' approach (call it Scheme C) can be constructed as a combination of the two approaches. The first approach (Scheme A) emphasizes flux reconstruction and the second approach (Scheme B) emphasizes solution reconstruction. The third approach would apply both forms of reconstruction. This approach is identical to Scheme B with the following modification: the viscous flux at the solution points (from step 3) is now corrected with a higher degree (k+1 order) correction function. This alleviates accuracy concerns associated with Scheme B's approach, while adding to the computational cost. This third approach (Scheme C) has been implemented and is numerically evaluated alongside the first and second approaches(see the examination of model diffusion problems presented later in this paper).


\vspace{0.2 in}
\begin{raggedleft}
{\bf Query 2:} Must FR schemes require a common flux with the form suggested by traditional DG schemes?
\end{raggedleft}
\vspace{0.2 in}

Recall that the forms of the common solution and solution gradient were constructed to match those of existing viscous discretizations in DG. The common solution and solution gradient took on a form consistent with that of BR2. This form allowed traditional DG methods to attain the expected k+1 order of accuracy. However, it is unclear that these forms are required for FR methods to attain the expected order of accuracy. Overall, FR is a more general approach which recovers a wider range of higher-order schemes. The treatment of common viscous fluxes in the FR framework need not be identical to the treatment of similar terms in the traditional DG framework.

\vspace{0.1 in}

\noindent Computational cost is an additional motivation for trying alternative approaches to the standard DG viscous construction. There is substantial additional computational cost associated with reconstructing the solution gradient at the boundaries of the cell. This cost will hinder the effectiveness of the method, particularly in attempts to extend it to higher dimensions. There are a number of possibilities for avoiding the reconstruction of the solution and the solution gradient. One possibility involves neglecting the reconstruction terms entirely. In this case, the solution and the solution gradient at the boundary would assume the average values of the solution and the solution gradient respectively. The common values of the solution and its gradient would assume the following forms (at the right boundary of element j),
%
\begin{equation}
\hat{u}^{\delta I}_R =  \frac {1}{2} \left ( \hat{u}^{\delta D}_R |_{j} + \hat{u}^{\delta D}_L |_{j+1}  \right),
\end{equation}
%
\begin{equation}
\left( \frac {\partial \hat{u}}{\partial r} \right)_{R}^{\delta I} =  \frac {1}{2} \left( \frac {\partial \hat{u}^{\delta D}_R |_{j}}{\partial r} +  \frac {\partial \hat{u}^{\delta D}_L |_{j+1}}{\partial r} \right) .
\end{equation}

\vspace{0.1in}
\noindent The common values (computed as shown above) can be utilized to compute the viscous flux in conjunction with the FR approach of Scheme A. This modified version of Scheme A will henceforth be referred to as Scheme D. This approach is evaluated in the examination of model diffusion problems which appear later in this paper.

\vspace{0.2 in}
\begin{raggedleft}
{\bf Query 3:} How do different choices of the correction functions effect the accuracy of the viscous flux?
\end{raggedleft}
\vspace{0.2 in}

The correction functions in flux reconstruction are tailored to promote the accuracy and stability of inviscid fluxes. The impact of these correction functions on the accuracy of the reconstructed solution and viscous fluxes is unclear. Fundamentally, these quantities have different stability and accuracy properties. It stands to reason that (due to their different stability properties) the viscous and inviscid fluxes place different constraints on their correction functions. There is not a clear physical reason to use the same correction function for both the inviscid and viscous fluxes. By the same token, there is not a clear physical reason to use the same correction function to reconstruct the solution. Nevertheless, there are clear practical reasons for both approaches. Chief amongst these are preserving the simplicity of the scheme and promoting its straightforward implementation within a computer code. Significant research has been devoted to generating admissible correction functions for the inviscid flux \cite{Vincent10}. This paper attempts to leverage these existing functions to treat the viscous flux and the solution. 

\vspace{0.1in}
\noindent In particular, this paper examines the accuracy associated with applying the new class of correction functions (ESFR functions) introduced in the previous section. These functions are parametrized by different values of c and thereafter used to correct both the solution and the viscous flux. The accuracy of the results are examined numerically in the context of model diffusion problems. Note: These results will appear in the final draft of the paper.


\vspace{0.2 in}
\section{Flux Reconstruction: Viscous case (2D)}

\noindent A discussion of correction functions for triangles and quads will appear in the final paper.
