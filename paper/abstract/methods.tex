%Purpose and scope; methods; key results; contributions to the state of the art; references
\section{Methods}
The high-order 3D Navier-Stokes (NS) solver for unstructured meshes developed in the Aerospace Computing Laboratory by Castonguay et al.\cite{castonguay2011} will be the main foundation for the simulations. To enable the use of Sub-Grid Scale (SGS) models in this code, the authors will implement the modeling used by Lodato et al.\cite{lodato2012}. The main advantage of using the high-order in-house code is its ability to run on Graphical-Processing-Units (GPUs)--thus reducing the running time of the simulations-- and to use either the flux-reconstruction approach, introduced by Huynh \cite{huynh2007}, or the unfiltered nodal Discontinuous-Galerkin (DG) approach, described in detail by Hesthaven et al.\cite{hesthaven2007}, at will. Additionally, this code enables the decoupling of the order of the solution representation and the order of the boundary geometry representation.

\subsection{Simulation setup}
The geometry matches the experimental setup of Lee et al.\cite{lee1997}. The Reynolds number, based on the diameter of the pipe from which the jet flows and the jet's bulk velocity, is Re = 23,000. Figure \ref{fig:setup} shows a slice of the 3D computational domain. $H$ is the distance beween the nozzle and the plate, $D$ is the diameter of the jet pipe, and $r$ is the distance along the surface between the center of the hemisphere and a point on the surface. $H/D = 6$ and $D/2R = 0.089$. Note that very likely the computational domain in this project will be much wider to avoid interaction of the side walls on the flow in the hemisphere.

\begin{figure}
\centering
\includegraphics[height=60mm]{./figures/ExpSetup.png} \\
\caption{Schematic of hemisphere computational domain used by Jefferson-Loveday et al.\cite{jefferson2010}}
\label{fig:setup}
\end{figure}

\subsection{Flux Reconstruction}
In the final draft, there will be a discussion about Flux Reconstruction and how it achieves high-order spatial discretization as shown by Williams et al.\cite{williams2011}.
\subsection{LES modeling approach}
The final draft will contain Lodato's\cite{lodato2012} SGS modeling formulation. We choose this formulation because it yielded promissing results when applied to the spectral difference method, another high-order method.
\subsubsection{GPU implementation}
This section, in the final draft, will mention how the authors ported the SGS modeling into the GPU arquitecture.
\subsection{Boundary polynomial representation}
The final draft will give a brief explanation, based on the paper by Wang \cite{wang2006}, of the high-order representation of the curved boundaries.
