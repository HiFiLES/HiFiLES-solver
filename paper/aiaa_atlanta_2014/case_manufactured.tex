% !TEX root = ./main.tex
\graphicspath{{figures_manufactured/}}% Set graphics path location


\subsection{Method of Manufactured Solutions}

This section describes the test of HiFiLES's spatial order of accuracy using the Method of Manufactured Solutions (MMS) in 2D and 3D for viscous flows. As shown by Salari et. al\cite{salari2000code}, the MMS test rigorously assesses the correctness of implementation of a solver of Partial Differential Equations. Simplex elements are crucial for simulations in unstructured meshes and have a more complex implementation than squares and hexahedra. As a result, we perform the MMS test in grids using simplex elements.

The MMS test for NS solvers requires checking the solver's solution against an exact solution. Such exact solution can be chosen arbitrarily. The NS equations can be satisfied with this arbitrary solution by including a time-dependent source term in the equations. Then, we solve

\begin{equation}
\frac{\partial U}{\partial t} +  \nabla \cdot {\bf F} = S
\end{equation}

For the following tests, we selected a smooth exact solution, so aliasing does not pollute the results. We picked

\begin{equation}\label{eq:NSwithSource}
U_{2D} = \l(
\begin{tabular}{c}
$\sin{(k(x+y) - \omega t)} + a$\\
$\sin{(k(x+y) - \omega t)} + a$\\
$\sin{(k(x+y) - \omega t)} + a$\\
$(\sin{(k(x+y) - \omega t)} + a)^2$
\end{tabular}
\r) \;\; 
U_{3D} = \l(
\begin{tabular}{c}
$\sin{(k(x+y+z) - \omega t)} + a$\\
$\sin{(k(x+y+z) - \omega t)} + a$\\
$\sin{(k(x+y+z) - \omega t)} + a$\\
$\sin{(k(x+y+z) - \omega t)} + a$\\
$(\sin{(k(x+y+z) - \omega t)} + a)^2$
\end{tabular}
\r)
\end{equation}

To find the value of $S$, we plug the values of our selected $U$ into the left-hand side of Equation~\eqref{eq:NSwithSource} and simplify. The resulting expression is $S$. 
We let Pr$=0.72, \gamma = 1.4, k = \pi, \omega = \pi, a = 3.0$ and $\mu = 0.001$.

The meshes used have dimensions $[-1,1] \times [-1,1]$ in 2D and $[-1,1] \times [-1,1] \times [-1,1]$ in 3D. Periodic boundary conditions were applied on the boundaries of the square and cube domains.

\begin{table}[htbp]
\centering
\begin{tabular}{ c c c c c c c} 
  
 Polynomial Order & Mesh: & 2x2x2 & 4x4x4 & 8x8x8 & 16x16x16 & Overall Order of Accuracy \\ 
 \hline 
 \multirow{2}{*}{$p = 1$} & $L_2$ error & 5.76e-01 & 1.35e-01 & 3.22e-02 & 7.90e-03 &   \\ 
  
   & $\mathcal{O}(L_2)$ &   & 2.10 & 2.06 & 2.03 & 2.06 \\ 
 \hline 
 \multirow{2}{*}{$p = 2$} & $L_2$ error & 4.09e-01 & 5.52e-02 & 6.87e-03 & 8.53e-04 &   \\ 
  
   & $\mathcal{O}(L_2)$ &   & 2.89 & 3.01 & 3.01 & 2.97 \\ 
 \hline 
 \multirow{2}{*}{$p = 3$} & $L_2$ error & 9.77e-02 & 5.97e-03 & 3.78e-04 &   &   \\ 
  
   & $\mathcal{O}(L_2)$ &   & 4.03 & 3.98 &   & 4.01 \\ 
 \hline 
 \multirow{2}{*}{$p = 4$} & $L_2$ error & 1.12e-02 & 6.39e-04 & 2.07e-05 &   &   \\ 
  
   & $\mathcal{O}(L_2)$ &   & 4.13 & 4.95 &   & 4.54 \\ 
 \hline 
 \multirow{2}{*}{$p = 5$} & $L_2$ error & 1.53e-01 & 5.08e-03 & 6.92e-05 &   &   \\ 
  
   & $\mathcal{O}(L_2)$ &   & 4.91 & 6.20 &   & 5.55 \\ 
 \hline 
 \end{tabular}
\caption{Accuracy of HiFiLES for NS equations with source term in tetrahedral meshes at $t = 10$. $L_2$ error is the $L_2$-norm of the error in the energy field: $\rho e$}
\label{table:tetsError1} 
 \end{table}

\begin{table}[htbp]
\centering
\begin{tabular}{ c c c c c c c c} 
  
 Polynomial Order & Mesh: & 4x4 & 8x8 & 16x16 & 32x32 & 64x64 & Overall Order of Accuracy \\ 
 \hline 
 \multirow{2}{*}{$p = 1$} & $L_2$ error & 7.92e-01 & 1.84e-01 & 4.36e-02 & 1.07e-02 & 2.68e-03 &   \\ 
  
   & $\mathcal{O}(L_2)$ &   & 2.10 & 2.08 & 2.03 & 2.00 & 2.05 \\ 
 \hline 
 \multirow{2}{*}{$p = 2$} & $L_2$ error & 1.29e-01 & 1.61e-02 & 1.95e-03 & 2.33e-04 & 2.86e-05 &   \\ 
  
   & $\mathcal{O}(L_2)$ &   & 3.00 & 3.05 & 3.06 & 3.03 & 3.04 \\ 
 \hline 
 \multirow{2}{*}{$p = 3$} & $L_2$ error & 1.01e-02 & 9.25e-04 & 5.71e-05 & 3.65e-06 & 2.35e-07 &   \\ 
  
   & $\mathcal{O}(L_2)$ &   & 3.45 & 4.02 & 3.97 & 3.96 & 3.88 \\ 
 \hline 
 \multirow{2}{*}{$p = 4$} & $L_2$ error & 2.60e-03 & 6.33e-05 & 2.00e-06 & 6.49e-08 & 3.62e-09 &   \\ 
  
   & $\mathcal{O}(L_2)$ &   & 5.36 & 4.98 & 4.95 & 4.16 & 4.88 \\ 
 \hline 
 \multirow{2}{*}{$p = 5$} & $L_2$ error & 7.15e-05 & 3.87e-06 & 6.31e-08 &   &   &   \\ 
  
   & $\mathcal{O}(L_2)$ &   & 4.21 & 5.94 &   &   & 5.07 \\ 
 \hline 
 \end{tabular}
\caption{Accuracy of HiFiLES for NS equations with source term in triangular meshes at $t = 1$. $L_2$ error is the $L_2$-norm of the error in the energy field: $\rho e$}
\label{table:trisError1} 
 \end{table}


\begin{table}[h]
\centering
\begin{tabular}{ c c c c c c c} 
  
 Mesh: &   & 2x2x2 & 4x4x4 & 8x8x8 & 16x16x16 & Overall Order \\ 
 \hline 
 \multirow{2}{*}{$p = 1$} & $L_2$ error & 1.98e+01 & 9.57e+00 & 4.55e+00 & 2.19e+00 &   \\ 
  
   & $\mathcal{O}(L_2)$ &   & 1.05 & 1.07 & 1.06 & 1.06 \\ 
 \hline 
 \multirow{2}{*}{$p = 2$} & $L_2$ error & 1.17e+01 & 2.98e+00 & 7.10e-01 & 1.71e-01 &   \\ 
  
   & $\mathcal{O}(L_2)$ &   & 1.97 & 2.07 & 2.06 & 2.03 \\ 
 \hline 
 \multirow{2}{*}{$p = 3$} & $L_2$ error & 3.17e+00 & 3.81e-01 & 4.73e-02 &   &   \\ 
  
   & $\mathcal{O}(L_2)$ &   & 3.06 & 3.01 &   & 3.03 \\ 
 \hline 
 \multirow{2}{*}{$p = 4$} & $L_2$ error & 5.21e-01 & 4.27e-02 & 2.69e-03 &   &   \\ 
  
   & $\mathcal{O}(L_2)$ &   & 3.61 & 3.99 &   & 3.80 \\ 
 \hline 
 \multirow{2}{*}{$p = 5$} & $L_2$ error & 3.20e+00 & 1.88e-01 & 4.79e-03 &   &   \\ 
  
   & $\mathcal{O}(L_2)$ &   & 4.09 & 5.29 &   & 4.69 \\ 
 \hline 
 \end{tabular}
\caption{Tets error2} 
 \end{table}

\begin{table}[htbp]
\centering
\begin{tabular}{ c c c c c c c c} 
  
 Polynomial Order & Mesh: & 4x4 & 8x8 & 16x16 & 32x32 & 64x64 & Overall Order of Accuracy \\ 
 \hline 
 \multirow{2}{*}{$p = 1$} & $L_2$ error & 1.61e+01 & 8.31e+00 & 3.81e+00 & 1.71e+00 & 7.84e-01 &   \\ 
  
   & $\mathcal{O}(L_2)$ &   & 0.96 & 1.12 & 1.15 & 1.13 & 1.10 \\ 
 \hline 
 \multirow{2}{*}{$p = 2$} & $L_2$ error & 4.05e+00 & 8.16e-01 & 1.90e-01 & 4.54e-02 & 1.11e-02 &   \\ 
  
   & $\mathcal{O}(L_2)$ &   & 2.31 & 2.11 & 2.06 & 2.04 & 2.12 \\ 
 \hline 
 \multirow{2}{*}{$p = 3$} & $L_2$ error & 4.71e-01 & 6.39e-02 & 7.03e-03 & 7.75e-04 & 8.84e-05 &   \\ 
  
   & $\mathcal{O}(L_2)$ &   & 2.88 & 3.18 & 3.18 & 3.13 & 3.11 \\ 
 \hline 
 \multirow{2}{*}{$p = 4$} & $L_2$ error & 1.01e-01 & 4.30e-03 & 2.31e-04 & 1.41e-05 &  &   \\ 
  
   & $\mathcal{O}(L_2)$ &   & 4.56 & 4.22 & 4.04 &  & 4.27 \\ 
 \hline 
 \multirow{2}{*}{$p = 5$} & $L_2$ error & 5.04e-03 & 2.50e-04 & 7.80e-06 &   &   &   \\ 
  
   & $\mathcal{O}(L_2)$ &   & 4.33 & 5.00 &   &   & 4.67 \\ 
 \hline 
 \end{tabular}
\caption{Accuracy of HiFiLES for NS equations with source term in triangular meshes at $t = 1$. $L_2$ error is the $L_2$-norm of the error in the gradient of the energy field:$\frac{\partial}{\partial x_i} (\rho e)$}
\label{table:trisError2} 
 \end{table}


%\begin{figure}
%\centering
%\includegraphics[height=35mm]{table_917} \\
%\caption{Accuracy of ESFR schemes for flow generated by a time-dependent source term on triangular grids, for the case of $p = 2$. The inviscid and viscous numerical fluxes were computed using a Rusanov flux with $\lambda = 1$ and a LDG flux with $\tau = 0.1$ and $\beta = \pm 0.5n$.}
%\label{fig:table_917}
%\end{figure}
%
%\begin{figure}
%\centering
%\includegraphics[height=35mm]{table_918} \\
%\caption{Accuracy of ESFR schemes for flow generated by a time-dependent source term on triangular grids, for the case of $p = 3$. The inviscid and viscous numerical fluxes were computed using a Rusanov flux with $\lambda = 1$ and a LDG flux with $\tau = 0.1$ and $\beta = \pm 0.5n$.}
%\label{fig:table_918}
%\end{figure}
%
%\begin{figure}
%\centering
%\includegraphics[height=20mm]{table_919} \\
%\caption{Explicit time-step limits ($\Delta t_{max}$) of ESFR schemes for flow generated by a time-dependent source term on the triangular grid with $\tilde{N} = 48$, for the cases of $p = 2$ and $3$. The inviscid and viscous numerical fluxes were computed using a Rusanov flux with $\lambda = 1$ and a LDG flux with $\tau = 0.1$ and $\beta = \pm 0.5n$.}
%\label{fig:table_919}
%\end{figure}
%
%\begin{figure}
%\centering
%\includegraphics[height=60mm]{figure_912} \\
%\caption{Contours of energy obtained using the ESFR scheme with $c = c_+$ and $\kappa = \kappa_+$ on the triangular grid with $\tilde{N} = 32$ for the case of $p = 3$. The inviscid and viscous numerical fluxes were computed using a Rusanov flux with $\lambda = 1$ and a LDG flux with $\tau = 0.1$ and $\beta = \pm 0.5n$.}
%\label{fig:figure_912}
%\end{figure}
%
%\begin{figure}
%\centering
%\includegraphics[height=35mm]{table_920} \\
%\caption{Accuracy of ESFR schemes for flow generated by a time-dependent source term on tetrahedral grids, for the case of $p = 2$. The inviscid and viscous numerical fluxes were computed using a Rusanov flux with $\lambda = 1$ and a LDG flux with $\tau = 0.1$ and $\beta = \pm 0.5n$.}
%\label{fig:table_920}
%\end{figure}
%
%\begin{figure}
%\centering
%\includegraphics[height=30mm]{table_921} \\
%\caption{Accuracy of ESFR schemes for flow generated by a time-dependent source term on tetrahedral grids, for the case of $p = 3$. The inviscid and viscous numerical fluxes were computed using a Rusanov flux with $\lambda = 1$ and a LDG flux with $\tau = 0.1$ and $\beta = \pm 0.5n$.}
%\label{fig:table_921}
%\end{figure}
%
%\begin{figure}
%\centering
%\includegraphics[height=15mm]{table_922} \\
%\caption{Explicit time-step limits ($\Delta t_{max}$) of ESFR schemes for flow generated by a time-dependent source term on the triangular grid with $\tilde{N} = 48$, for the cases of $p = 2 and 3$. The inviscid and viscous numerical fluxes were computed using a Rusanov flux with $\lambda = 1$ and a LDG flux with $\tau = 0.1$ and $\beta = \pm 0.5n$.}
%\label{fig:table_922}
%\end{figure}
%
%\newpage
%\begin{figure}
%\centering
%\includegraphics[height=60mm]{figure_913} \\
%\caption{Contours of energy obtained using the ESFR scheme with $c = c_+$ and $\kappa = \kappa_+$ on the tetrahedral grid with $\tilde{N} = 32$ for the case of $p = 3$. The inviscid and viscous numerical fluxes were computed using a Rusanov flux with $\lambda = 1$ and a LDG flux with $\tau = 0.1$ and $\beta = \pm 0.5n$.}
%\label{fig:figure_913}
%\end{figure}
